\documentclass{ds-report}
\assignment{Java EE} % Set to `Java RMI`, `Java EE` or `Google App Engine`.
\authorOne{Dries Janse} % Name of first team partner.
\studentnumberOne{r0627054} % Student number of first team partner.
\authorTwo{Steven Ghekiere} % Name of second team partner.
\studentnumberTwo{r0626062}  % Student number of second team partner.


\begin{document}
	\maketitle

	\paragraph{1. Outline the different tiers of your application, and indicate where classes are located.} \mbox{}\\\\




	
	\paragraph{2. Why are client and manager session beans stateful and stateless respectively?} \mbox{}\\\\




	\paragraph{3. How does dependency injection compare to the RMI registry of the RMI assignment?} \mbox{}\\\\




	\paragraph{4. JPQL persistence queries without application logic are the recommended approach for retrieving rental statistics. Can you explain why this is more effcient?} \mbox{}\\\\





	\paragraph{5. How does your solution compare with the Java RMI assignment in terms of resilience against server crashes?} \mbox{}\\\\





	\paragraph{6. How does the Java EE middleware reduce the effort of migrating to another database engine?} \mbox{}\\\\





 

	\paragraph{7. How does your solution to concurrency prevent race conditions?} \mbox{}\\\\






	\paragraph{8. How do transactions compare to synchronization in Java RMI in terms of the scalability of your application?} \mbox{}\\\\









	\paragraph{9. How do you ensure that only users that have specifcally been assigned a manager role can open a ManagerSession and access the manager functionality?} \mbox{}\\\\








	\paragraph{10.  Why would someone choose a Java EE solution over a regular Java SE application with Java RMI?} \mbox{}\\\\



	
	\clearpage

	% You can include diagrams here.
	
\end{document}